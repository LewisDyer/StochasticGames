
\chapter{Background (3 pages)}

Give context on existing work, both in model checking and in automated game design (RMT coursework very helpful for this). Identify a gap in current work (e.g certain types of games that haven't been analysed very much, or discussion on robustness of models). How can I tie this into my project?

Note: don't go into specific details about model types here (MDPs/POMDPs/CSGs) - leave them to later on.

The very key points I need to mention:

\begin{itemize}
    \item Probabilistic model checking
    \item PRISM specifics
    \item Property specification
    \item Stochastic games
    \item Previous work in automated game balancing
    \item Gap in current work (e.g focus on dice/hidden information/weighted dice?)
\end{itemize}

In order to begin formal analysis of a game, we must first formally define a model of a game, along with properties of this game we wish to consider.

\section{Discrete-time Markov chains}
\label{back:stoc_game}

We first introduce a discrete-time Markov chain (or DTMC), as described by Kwiatkowska et al. in \cite{kwiatkowska_stochastic_2007}. This model is refined and adapted in each case study to develop a different type of model suited to different types of games.

\begin{definition}
\label{back:dtmc}
    Given a set of atomic propositions, denoted $AP$, a discrete-time Markov chain (DTMC) is defined as a tuple $\mathcal{D} = (S, \bar{s}, \mathbf{P}, L)$ such that:

    \begin{itemize}
        \item $S$ is a finite set of states;
        \item $\bar{s}$ is the initial state for the DTMC;
        \item $\mathbf{P} : S \times S \rightarrow [0,1]$ is the \emph{transition probability matrix}, where $P(s, s')$ is the probability of transitioning from state $s$ to state $s'$, such that the sum of probabilities of outgoing transitions from each state is $1$.
        \item $L: S \rightarrow 2^{AP}$ is a \emph{labelling function}, where for $s \in S$ a state, $L(s)$ denotes the subset of atomic propositions which hold at that state.
    \end{itemize}

\end{definition}


From this definition of a DTMC, we also define the states where an atomic proposition holds.

\begin{definition}
\label{back:sat}

    For $a \in AP$ and a set of states $S$ as in Definition \ref{back:dtmc}, denote $Sat(a) = \{s \in S | a \in L(s)\}$.

\end{definition}


As a motivating example, we define a classic board game using a DTMC:

\begin{example}
\label{back:chess}

    A game of chess can be defined as a DTMC from Definition \ref{back:dtmc} as follows:

    \begin{itemize}
        \item $S$ is the set of states for the game, representing all possible board positions throughout a game of chess, along with the current player's turn (since a given board position could be obtained where either black or white is the current player).
        \item $\bar{s}$ is the initial board position of a standard game of chess, where white makes the first move.
        \item $\mathbf{P}$ assigns each valid move a probability of being chosen at a particular board position. More precisely, a transition to a board position is only valid if the transition can be presented by one legal move according to the rules of chess (for instance, the white player can only move white pieces, can only move one piece per move with the exception of castling, and cannot move a piece if the black player can capture the white king in the following turn).
        \item $L$ labels each state with a set of propositions which are true at that state. For instance, if we let $\mathbf{C_w}$ be the atomic proposition that the white player is in check in a given state, then for $s \in S$ a state, $\mathbf{C_w} \in L(s)$ means that the white player is in check in state $s$.
    \end{itemize}

    In this model, the key aspect in determining the skill level of a player is in the transition probability matrix. In theory, there is always an optimal move to make in chess, so for each board position, an optimal player's transition probability matrix will return $1$ for precisely one action, and return $0$ for all other actions. But in practice chess is far too complex for players, human or computer alike, to compute this optimal strategy. Hence strategies may be probabilistic, such as a chess player who occasionally "blunders", making a move which places them in a disadvantageous position.

\end{example}

Playing a game is represented as an infinite \emph{path}, denoted as a sequence $\omega = s_0 \rightarrow s_1 \rightarrow \dots$ where $\mathbf{P}(s_k, s_{k+1})>0$ for all $k\geq0$, or in other words where each action is possible. Note that in states where a game terminates (for instance, states where either the white or black player is in checkmate, or where the game ends by stalemate), we typically define a self-loop in each of these states with probability 1 to ensure the path remains infinite.

We may also augment this DTMC with a set of \emph{reward structures}, which are each comprised of a \emph{state reward function} $\rho : S \rightarrow \mathbb{N}$, associating each state with the value of a reward, and a \emph{state transition function} $\iota : S \times S \rightarrow \mathbb{N}$ which associates each transition with the value of a reward. For instance, in chess, we may define a reward structure where $\rho$ returns $0$ for all states, and $\iota$ returns $1$ for all transitions where the white player captures a black piece, and returns $-1$ for all transitions where the black player captures a white piece. Note that the reward values may be continuous (though they may never be negative), but we only consider reward values in $\mathbb{Z}$. These reward values can be utilised and examined in various different ways to evaluate a DTMC. More detail on this is provided in \ref{back:prop_spec}, but in particular the expected value of a reward value over all possible paths through a DTMC is frequently considered during model checking.

\section{Property specification}
\label{back:prop_spec}

When analysing probabilistic models, including DTMCs, we consider two main types of properties: \emph{probabilistic reachability} properties and \emph{reward-based} properties. We may define properties in terms of \emph{Probabilistic Computation Tree Logic} (PCTL), as defined by Hansson and Jonsson in \cite{hansson_logic_1994}. These properties are comprised of path quantifiers and temporal operators, though for the purposes of this dissertation we only consider one temporal operator. In particular, the future operator, denoted $\mathbf{F}$, considers whether a particular proposition \emph{eventually} always holds after some state on a given path.

A closely related operator is the until operator, denoted  $\mathbf{U}$, which considers two propositions, say $a$ and $b$, and determines whether for a given path, $a$ always holds before some state, then $b$ holds for the given state. Indeed, the future operator is actually equivalent to the until operator when $a$ is the proposition which always holds, denoted \verb+true+. We use this equivalence in \ref{back:prob_mod_check} in order to perform model checking on probabilistic reachability properties.

For probabilistic reachability properties, we also include the $\mathbf{P}$ operator, which considers the probability of a particular property holding, including its proposition and any temporal operators, across all possible executions of the TSG. For instance, using Example \ref{back:chess}, if we define $\mathtt{white\_win}$ as the proposition that white is considered to have won the game in the current state, the property $\mathbf{P}_{=?} [\mathbf{F}\ \mathtt{white\_wins}]$ represents the probability that white eventually wins the game of chess.

For reward-based properties, PRISM defines an extension to PCTL introducing the $\mathbf{R}$ operator, which allows for properties where the value of a reward is taken into account. For the purposes of this dissertation we only consider one type of reward-based property, namely the \emph{reachability reward} property, referring to the expected cumulative value of a reward along a path, until a state satisfying a particular proposition is reached. As an example, again building on Example \ref{back:chess}, using the previously defined reward structure, the property $\mathbf{R}_{=?} [\mathbf{F}\ \mathtt{white\_win}]$ represents the expected value of the difference between the number of pieces the white and black player control in the cases where the white player eventually wins. These reward-based properties can provide useful information when balancing games. For instance, we may initially expect that the diffence in pieces captured is a useful measure of how "close" a game of chess is, and this measure could be evaluated via model checking, where we may find it insufficient as a balance measure. For instance, this measure gives equal weighting to every piece, whereas some pieces such as queens are often considered more valuable than other pieces.

\section{Probabilistic model checking}
\label{back:prob_mod_check}

% This is good, but definitely more like introduction/motivation

%As we have established, many games exhibit probabilistic behaviour. For instance, board games frequently make use of dice in order to introduce random elements into a game, in order to improve the game's replayability. Moreover, the strategies employed while playing these games are often probabilistic. For instance, games with a high degree of non-transitivity encourage probabilstic strategies, since the lack of optimal action means that non-probabilistic strategies are relatively simple to exploit. To give an example, in the game of rock paper scissors, the strategy of always playing scissors will always lose against the strategy of always playing rock.

%As a result of this, typical verification techniques relying on random samples, such as Monte Carlo methods, can be unreliable when considering properties of stochastic games. Statistical methods can only ever provide \emph{probabilistic} guarantees about the game, since the series of samples taken may be entirely unrepresentative of typical gameplay. Moreover, when considering the optimal behaviour of a game, we frequently encounter cases of nondeterminism, where a choice must be made. Most statistical techniques currently resolve nondeterminism uniformly, which amounts to making choices at random, without any particular rationale or analysis behind these choices. Hence we require a more sophisticated class of techniques for analysing stochastic games.


Now that we have defined a DTMC, along with defining properties of that DTMC to consider, we now define probabilistic model checking - the class of methods for verifying the probability of some event occurring during the process of playing a game. Compared to non-probabilistic model checking, this presents some unique challenges, such as in the following example:

\begin{example}
\label{back:chess-reachability}
    Consider a game of chess, defined in Example \ref{back:chess}, and define $\mathbf{M_w}$ and $\mathbf{M_b}$ as the atomic propositions that the white player and black player, respectively, are in checkmate in a particular position. In non-probabilistic model checking, we may consider the reachability of $\mathbf{M_w}$ - in other words, we consider whether we can always reach a state where the white player is in checkmate, regardless of which reachable board position we start in.
    
    In particular this result is binary - either we can always reach a state where $\mathbf{M_w}$ holds, or there exists a board position where white can never lose via checkmate. Indeed, in chess the latter holds - for instance, if the black player only has a king remaining, then there is no way for white to lose via checkmate, since black has insufficient material. And since our model is essentially a directed graph, we can employ standard widely-known algorithms such as a breadth first search in order to verify such reachability properties.

    However, if we consider probabilistic model checking, then as well as quantitative properties, such as the above, we can also consider quantitative properties, such as the \emph{probability} that the white player is in checkmate from a particular board position. In PCTL, this may be expressed as the property $\mathbf{P}_{=?} [\mathbf{F} \, \mathbf{M_w}]$. This is a far more general and powerful property - for instance, an advanced chess player, who has found themselves in a state such that they are in disadvantage over the other player, may seek to reach states where the game is unlikely to end in checkmate, maximising the probability of a draw. However calculating this probability is more complex than for non-probabilistic reachability problems, since we must also consider every possible path that can be taken in order to reach a particular state.

\end{example}

Evaluating probabilistic reachability properties is a recursive process, since we need to consider the probabilistic reachability of each state that a given state can reach. Hence we need to define the "base cases" of this recursion, which are described below as in \cite{kwiatkowska_stochastic_2007}, with slightly different notation:

\begin{definition}
\label{back:S_yes}

    For a DTMC $\mathcal{D}$, with a set of states $S$ and a given atomic proposition $a$, we can partition the state space into three subsets as so:

    \begin{itemize}
        
        \item $S^{yes} = Sat(\mathbb{P}_{\geq 1}[F \: a])$, the set of all states where $a$ eventually holds with probability 1.
        \item $S^{no} = Sat(\mathbb{P}_{\leq 0}[F \: a])$, the set of all states where $a$ never holds in any subsequent reachable state.
        \item $S^{?} = S \setminus (S_{yes} \cup S_{no})$ the set of all states where the probability of $a$ holding eventually is neither 0 nor 1.

    \end{itemize}

\end{definition}

The sets $S^{yes}$ and $S^{no}$ are fairly straightforward to compute using graph-based algorithms. The precise details of their construction are described further in \cite{kwiatkowska_stochastic_2007}, but $S^{no}$ is constructed via a simple reverse traversal of the DTMC, starting from $Sat(a)$ and obtaining all states that can reach a state in $Sat(a)$, then removing these states from $S$. $S^{yes}$ is constructed very similarly, except this construction starts with $S^{no}$ and traverses the DTMC to find all states where the probability of reaching a state in $Sat{a}$ is less than 1, then removing these states from $S$.

After this precomputation process, we can now calculate the probability of a probabilistic reachability property holding across a set of states:

\begin{definition}
\label{back:F_operator}

    For a DTMC $\mathcal{D}$, with a set of states $S$, a partition of $S$ as in Definition \ref{back:S_yes}, and an atomic proposition $a$, we denote $Prob^{\mathcal{D}}(s, \mathbf{F} \: a)$ as the probability that an execution of $\mathcal{D}$ starting at state $s$ eventually reaches a state where $a$ holds. Furthermore, we have for any $s \in S$ that:

    \begin{equation*}
        Prob^{\mathcal{D}}(s, \mathbf{F} \: \Phi) = \sum_{s' \in S^{?}} \mathbf{P}(s, s') \cdot Prob^{\mathcal{D}}(s, \mathbf{F} \: \Phi) \: + \sum_{s' \in S^{yes}} \mathbf{P}(s, s')
    \end{equation*}
    
\end{definition}

From Definition \ref{back:F_operator}, calculating reachability probabilities requires solving a system of linear equations containing $|S^{?}|$ unknowns. There are many different methods for solving these equations, but in practice iterative methods (such as the Gauss-Siedel method) are preferred to direct methods (such as power iteration) - iterative methods generally do not give exact solutions, opting for values which eventually converge to an exact solution, but direct methods can take infeasibly many steps to compute an exact solution.

\section{The PRISM model checker}
\label{back:PRISM}

PRISM \cite{kwiatkowska_prism_2011} is a software tool which implements probabilistic model checking on a wide variety of models. In particular, PRISM defines two languages in order to facilitate model checking: the \emph{PRISM language} for formally defining models, along with the \emph{PRISM property specification language} for formally defining PCTL properties (or rather, an extension of PCTL adding reward-based properties).

\subsection{The PRISM modelling language}
\label{back:PRISM-modelling}

PRISM provides a modelling language in order to formally specify probabilistic models. In particular, many probabilistic models exhibit \emph{emergent complexity}, where simple descriptions of behaviour can lead to complex models. Hence, a modelling language is essential to allow describing the behaviour of a model, rather than a more complicated model specification. Note that we do not provide a full description of the PRISM modelling language here, instead focusing on the key features used in the various case studies - more details are available in the PRISM manual \cite{noauthor_prism_nodate}, along with additional examples.

PRISM models are defined as a set of \emph{modules}, where each module consists of a set of \emph{variables} and a set of \emph{commands}. Variables may be either integers or boolean values, and in particular integers must be bounded, since otherwise the state space of a model is unbounded. Variables can be accessed by any module, but modules can only alter their own variables.

The key structure of a PRISM model is a \emph{command}, of the following form:

\begin{verbatim}
    [] guard -> prob_1 : update_1 + ... + prob_n : update_n;
\end{verbatim}

A \emph{guard} represents a proposition, identifying the set of states where the command is applicable. The command here has $n$ transitions, with a probability of each transition occurring such that the probabilities sum to $1$. Each update is comprised of a set of variable updates. And each possible combination of values for each variable represents a state of the resulting model, so a set of variable updates represents a transition between two states.

Note that guards may overlap, including between modules. This behaviour is handled differently depending on the type of model being constructed. For a DTMC each command which is enabled has an equal probability of being selected, but for other model types overlapping guards induce nondeterministic choice - a key aspect of defining strategies for games, as described further in the case studies.

In some cases, there may be a desire for modules to synchronise commands - for instance, two players making a decision simultaneously. This behaviour is supported via the use of \emph{action labels}. As an example, consider the following PRISM command:

\begin{verbatim}
    [cover_1] b1=0 -> (b1'=1);
\end{verbatim}

Here, very little logic is given for when the variable \verb+b1+ is updated. However, the \verb+cover_1+ action label allows this command to synchronise with another command with the same action label in another module. This is especially useful for defining "updater" modules - modules which control a particular variable and perform updates, but make no contribution towards the decision making of a particular model. We also note that action labels may be used even where no explicit synchronisation takes place. Indeed, action labels are frequently used in order to describe an execution of a model, which is particularly useful when debugging a model via manual simulation.

For convenience, PRISM models also define \emph{formulas} and \emph{labels}. Formulas act as shorthand for a particular expression, allowing for a complex expressed to be referred to on multiple occasions without needless code duplication. A simple example of a formula is given below:

\begin{verbatim}
    formula score = b1*1 + b2*2 + b3*3 + b4*4 + b5*5 + b6*6;
\end{verbatim}

Labels serve a very similar purpose, but they are boolean expressions which define a set of states satisfying a particular proposition, which are frequently used when specifying properties. For instance, the following label shows how labels can be utilised to define a particular characteristic of states of a model:

\begin{verbatim}
    label "game_over" = (phase>=6);
\end{verbatim}

Finally, PRISM allows for specifying reward structures. Reward structures comprise of a guard specifying when the reward is applicable, along with the value of a reward when this guard is satisfied, as in the following example:


\begin{verbatim}
    rewards "no_rolls"
        state=0: 1;
    endrewards
\end{verbatim}

Note that PRISM supports two main types of rewards - instantaneous rewards, considering the value of a reward in a specific state, and cumulative rewards considering the total value of a reward throughout an execution of the model. This distinction is applied at the property specification stage, rather than model construction.

\subsection{The PRISM property specification language}
\label{back:PRISM-prop}

When specifying properties of a model to be evaluated, PRISM defines a property specification language which supports and extends a variety of probabilistic temporal logics, although for the purposes of this dissertation we only consider the extension of PCTL with reward-based properties, as described in \ref{back:prop_spec}. As a simple example, consider the following property in PRISM:

\begin{verbatim}
    P=? [ F game_over&score=10 ]
\end{verbatim}

The main difference between the PRISM modelling language and "pure" PCTL syntax is the definition of the atomic proposition. Here, we refer to labels and variables from within a PRISM module in order to define an atomic proposition, allowing us to easily define a set of states where this proposition holds, namely where the game is over and the current score is 10. Reward-based properties can also be defined in a similar manner:

\begin{verbatim}
    R{"total_boards"}=? [ F game_over ]
\end{verbatim}

Note that PRISM models may contain many possible reward structures, so we specify the particular reward structure considered here using braces.

The PRISM modelling language provides a convenient way to express properties in PCTL, but one particular syntactic benefit of the PRISM modelling language is the use of undefined constants. For instance, consider adapting the above probabilistic reachability property in order to use an undefined constant for the score:

\begin{verbatim}
    const int k;
    P=? [ F game_over&score=k ]
\end{verbatim}

When model checking is performed, an \emph{experiment} can be defined, where different values of \verb+k+ may be considered. For instance, if the value of \verb+score+ is bounded, then allowing \verb+k+ to vary over all possible values of \verb+score+ can be utilised in order to produce a probability distribution of \verb+score+ when the model terminates.

%% move this to games section later on
% In addition, later on in the case studies, we define nondeterministic models, along with coalitions. Hence, properties may look more similar to this:
% 
% \begin{verbatim}
%     <<p1>>Pmax=? [ F game_over&score=k ]
% \end{verbatim}
% 
% In this property, \verb+<<p1>>+ denotes the set of players in the \emph{coalition} - in other words, players whose actions may be directly controlled. Moreover, \verb+Pmax+ means that the maximum probability is considered \emph{under all possible resolutions of nondeterminism}. In particular, care must be taken when computing conditional probabilities, since in general the decisions made in order to obtain the lower bound and the upper bounds of a particular probability are rather different. 

\section{Related work}

% just copying over RMT for now, will probably edit a fair bit

\label{back:related}

